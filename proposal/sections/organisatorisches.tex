%% LaTeX2e class for student theses
%% sections/conclusion.tex
%% 
%% Karlsruhe Institute of Technology
%% Institute for Program Structures and Data Organization
%% Chair for Software Design and Quality (SDQ)
%%
%% Dr.-Ing. Erik Burger
%% burger@kit.edu
%%
%% Version 1.1, 2014-11-21

\chapter{Organisatorisches}
\label{ch:Organisatorisches}
Im folgenden Abschnitt wird beschrieben, wie die geplante Arbeit ablaufen soll.
\section{Betreuer}
\textbf{Erstgutachter}: Prof. Dr. Ralf H. Reussner \\
\textbf{Zweitgutachter}: Jun.-Prof. Dr.-Ing. Anne Koziolek \\
\textbf{Erster Betreuer}: M. Sc. Stephan Seifermann \\
\textbf{Zweiter Betreuer}: Dipl.-Inform. Emre Taşpolatoğlu 

\section{Artefakte}
Im Laufe der Bachelorarbeit werden die folgenden Artefakte erzeugt werden:
\begin{itemize}
\item Anforderungen an Modellierungsumfang aus Vertraulichkeitsdefinitionen
\item Erweiterungskonzept für Meta-Modelle der einzelnen Sichten 
\item Erweiterungskonzept der Meta-Modelel für Palladio 
\item Datenflussmodellierung für Palladio mithilfe einer Meta-Modell-Erweiterung  
\item Ausarbeitung der Bachelorarbeit 
\item Vortrag zur Bachelorarbeit
\end{itemize}

\section{Entwicklungsumgebung und Werkzeuge}
Die folgenden Entwicklungsumgebungen und Werkzeuge werden für die Bachelorarbeit benötigt:
\begin{itemize}
\item TexMaker
\item Eclipse Mars
\item Palladio 4.0
\item MS Visio
\end{itemize}